% -*- mode:flyspell; mode:latex -*-
% \documentclass[a4paper,twoside,11pt]{book}
\documentclass[12pt]{article}
\usepackage[latin1]{inputenc}
\usepackage[T1]{fontenc}
\usepackage[english]{babel}
\usepackage{graphicx}
\usepackage{float}
\usepackage{amsmath}
\usepackage{hyperref}
% \usepackage{subfigure}
\usepackage{color}
\usepackage[all]{hypcap}
% \usepackage[normalem]{ulem}  % for striking out
% \usepackage{fancyhdr}
% \pagestyle{fancy}
% \fancyhead[C]{}
% \fancyhead[L] {\it{Mu2e-doc-4555-v1.0} }

\usepackage[square,sort,comma,numbers]{natbib}

% \usepackage[backend=biber, style=numeric-comp, sorting=ynt] {biblatex}

% \addbibresource{mu2e_internal_notes.bib}
% \addbibresource{radiative_muon_capture.bib}
% \addbibresource{radiative_pion_capture.bib}

\newcommand {\red}          {\color{red}}
\newcommand {\blue}         {\color{blue}}
\newcommand {\mubunch}      {$\mu$Bunch}
\newcommand {\ra}           {\mbox{$\rightarrow$}}
\newcommand {\MuMinusEPlus} {\mbox{$\mu^- \rightarrow e^+$}}


\newcommand*{\plogo}{\includegraphics[scale=6]{\mchiappiFigures/mu2e_logo_oval}}

%\newcommand*{\plogo}{\includegraphics[scale=6]{mu2e_logo_oval}}

\begin{document}

\begin{titlepage}
  \begin{flushright}
    \bf {MU2E/PHYSICS/22262} \\
    version 0.0
    \today
  \end{flushright}

  \vspace{1cm}
  
  \begin{center}
    {\Large \bf Understanding the kMax fits of the RMC spectra} 
    
    \vspace{1cm}
    
    E.Diociaiuti (Roma), M.Mackenzie (Northwestern), P.Murat(FNAL)
    
    % \footnote{\texttt{Fermilab; e-mail: murat@fnal.gov}}
    \vspace{0.3cm}
    
    \vspace{0.8cm}                           
  \end{center}

  \begin{abstract}
    
    This note summarizes our effort on understanding the kMax fits of the RMC spectra.

    Mistakes and other internal inconsistencies found in the literature make
    it difficult to rely on the results of other experiments when predicting RMC
    background to the search of mu- -->e+ conversion process.
    
    To reliably predict RMC background to mu- --> e+, Mu2e needs to have its own
    measurement of the endpoint of the RMC photon spectrum on Al.
    
  \end{abstract}

\end{titlepage}
% \frontmatter
% \chapter*{Abstract}
%
% \addcontentsline{toc}{chapter}{Abstract}
%
% \mainmatter
%
{\tableofcontents}

%%%%%%%%%%%%%%%%%%%%%%%%%%%%%%%%%%%%%%%%%%%%%%%%%%%%%%%%%%%%%%%%%%%%%%%%%%%%%%%
%\chapter{Calibration}
%%%%%%%%%%%%%%%%%%%%%%%%%%%%%%%%%%%%%%%%%%%%%%%%%%%%%%%%%%%%%%%%%%%%%%%%%%%%%%%
\section{ RMC as a background to a --> b }
% 
% \begin{figure}[H]
%   \includegraphics[width=1.05\textwidth]{figures/png/beam_flash_electron_momentum} 
% 
%   \caption{
%    momentum spectrum of the beam flash electrons
%   }
% \end{figure}

Radiative muon capture physics is one of the most important backgrounds to the search
of \MuMinusEPlus\ conversion. Given the proximity between the expected signal energy,
92.32 MeV, and the published endpoint of the RMC spectrum on aluminum, $90 \pm 2$ MeV
\cite{RMC_1999_PhysRevC.59.2853}, variation of the RMC spectrum endpoint within the
published errors could result in significant changes in the expected RMC background.

References \cite{RMC_1992_PhysRevC.46.1094}, \cite{RMC_1999_PhysRevC.59.2853} provide
sufficient information for the readers to reproduce the published fits of the
RMC spectra.

Discuss the closure approximation

We use published parameterizations of the TRIUMF RMC spectrometer response
to fit the experimental data and determine kMax values for different nuclei
as well as the corresponding fit uncertainties.

We compare our fit results to the published ones.

%%%%%%%%%%%%%%%%%%%%%%%%%%%%%%%%%%%%%%%%%%%%%%%%%%%%%%%%%%%%%%%%%%%%%%%%%%%%%%%
\section { Parameterization of the Detector Response}


Published are the measured data. To compare them to the model predictions,
one needs to know how the theoretical spectrum is modified by the detector
response, i.e. know the detector response.

Fortunately, references  \cite{RMC_1992_PhysRevC.46.1094}, \cite{RMC_1999_PhysRevC.59.2853}
have the detector response published.

Also published is the RPC spectrum on $H_2$ used for calibration.

We check how well different parameterizations of the detector response describe the
calibration peak.

%%%%%%%%%%%%%%%%%%%%%%%%%%%%%%%%%%%%%%%%%%%%%%%%%%%%%%%%%%%%%%%%%%%%%%%%%%%%%%%
\section { Input Data for fits}

Input data used are the digitized figures from references
\cite{RMC_1992_PhysRevC.46.1094,RMC_1999_PhysRevC.59.2853}.

Checks of the digitization accuracy

Errors - statistical. Systematic uncertainty on the energy scale in the
published data - less than 0.5 MeV.

%%%%%%%%%%%%%%%%%%%%%%%%%%%%%%%%%%%%%%%%%%%%%%%%%%%%%%%%%%%%%%%%%%%%%%%%%%%%%%%
\section { Fits }


%%%%%%%%%%%%%%%%%%%%%%%%%%%%%%%%%%%%%%%%%%%%%%%%%%%%%%%%%%%%%%%%%%%%%%%%%%%%%%
\subsection { Fits of the 1992 data }

%%%%%%%%%%%%%%%%%%%%%%%%%%%%%%%%%%%%%%%%%%%%%%%%%%%%%%%%%%%%%%%%%%%%%%%%%%%%%%
\subsection { Fits of the 1995 data }

%%%%%%%%%%%%%%%%%%%%%%%%%%%%%%%%%%%%%%%%%%%%%%%%%%%%%%%%%%%%%%%%%%%%%%%%%%%%%%
\subsection { Fits of the 1998 data }

%%%%%%%%%%%%%%%%%%%%%%%%%%%%%%%%%%%%%%%%%%%%%%%%%%%%%%%%%%%%%%%%%%%%%%%%%%%%%%
\subsection { Fits of the 1999 data }

%%%%%%%%%%%%%%%%%%%%%%%%%%%%%%%%%%%%%%%%%%%%%%%%%%%%%%%%%%%%%%%%%%%%%%%%%%%%%%%
\section { Discussion }


1. Our fits result in significantly smaller errors on kMax.

2. '1999 parameterization of the detector response gives significantly better
    description of all datasets published by the TRIUMF RMC spectrometer
    ('1992, '1995, '1998, and '1999)

    3. Fit chi2's using '1992 response are large enough to suggest wrong response.

    4. '1992 response is conssitent with the RPC peak, while '1999 response - is not

    5. different kMax fits for the same target vary on a scale large compared to the
    fit errors

    6. how our fit results are different from the published? 


% \newpage
%%%%%%%%%%%%%%%%%%%%%%%%%%%%%%%%%%%%%%%%%%%%%%%%%%%%%%%%%%%%%%%%%%%%%%%%%%%%%%%
\section{ Summary }


The only thing we can claim for certain is that the fit uncertainties on kMax are
of the order of 0.5 MeV, not 2 MeV. Published uncertainties of about 2-3 MeV
have been derived using wrong procedure (fit of the distribution of chi2/dof
instead of the total chi2)


For Al, the fit results are consistent with the published - about 90 MeV.


Plan : use aluminum data

a)Assume $90 \pm 0.5$ MeV, determine the RMC background

b) assume kMax + 3 sigma, determine the RMC background

%%%%%%%%%%%%%%%%%%%%%%%%%%%%%%%%%%%%%%%%%%%%%%%%%%%%%%%%%%%%%%%%%%%%%%%%%%%%%%% 
\section{ Acknowledgements }

We want to thank ...

%%%%%%%%%%%%%%%%%%%%%%%%%%%%%%%%%%%%%%%%%%%%%%%%%%%%%%%%%%%%%%%%%%%%%%%%%%%%%%
     %% 
     %% % \addcontentsline{toc}{chapter}{Bibliography}
     %
\bibliographystyle{plain}
\bibliography{radiative_muon_capture}

% \printbibliography
\end{document}


% ------------------------------------------------------------------------------
% templates
% ------------------------------------------------------------------------------
% Table ~\ref{table:summary} gives summary the numbers used in this study.
% 
% \hspace{-0.1in}
% \begin{table}[htbp]
%   \label{table:summary}
%   \begin{center} 
%     {\renewcommand{\arraystretch}{1.0}   % change 1.0 to 1.1 to increase the spacing between the table lines
%       \begin{tabular}{|c|c|c|c|}
%         \hline
%                             & default TS geometry & misaligned TS geometry   &  Ratio(default/misaligned)    \\ 
%         \hline
%         $N_{POT}$            &  $4.96 \cdot 10^6$  &    $5.00 \cdot 10^6$      &   0.992   \\ 
%         $N_{\mu}^{TS3u}$      &  65648              &     61354                 &   1.070   \\ 
%         $N_{\mu}^{TS5}$       &  28517              &     27351                 &   1.043   \\ 
%         $N_{\mu}^{ST}$        &  8868               &      8396                 &   1.056   \\ 
%         $N_{\mu}^{ST}/N_{POT}$ &  $1.79 \pm 0.02$    &    $1.68 \pm 0.02$        &   $1.065 \pm 0.03$        \\ 
%         \hline
%       \end{tabular}
%     }
%   \end{center}
%   \caption{
%     Muons rates at different points of the Mu2e beamline and stopping muon rates for nominal and 
%     misaligned TS geometries
%   }
%   % \vspace{0.5in}
% \end{table}
